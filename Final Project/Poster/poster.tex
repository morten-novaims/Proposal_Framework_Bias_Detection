%%%%%%%%%%%%%%%%%%%%%%%%%%%%%%%%%%%%%%%%%
% Jacobs Landscape Poster
% LaTeX Template
% Version 1.0 (29/03/13)
%
% Created by:
% Computational Physics and Biophysics Group, Jacobs University
% https://teamwork.jacobs-university.de:8443/confluence/display/CoPandBiG/LaTeX+Poster
% 
% Further modified by:
% Nathaniel Johnston (nathaniel@njohnston.ca)
%
% This template has been downloaded from:
% http://www.LaTeXTemplates.com
%
% License:
% CC BY-NC-SA 3.0 (http://creativecommons.org/licenses/by-nc-sa/3.0/)
%
%%%%%%%%%%%%%%%%%%%%%%%%%%%%%%%%%%%%%%%%%

%----------------------------------------------------------------------------------------
%	PACKAGES AND OTHER DOCUMENT CONFIGURATIONS
%----------------------------------------------------------------------------------------

\documentclass[final]{beamer}

\usepackage[scale=1.24]{beamerposter} % Use the beamerposter package for laying out the poster

\usetheme{confposter} % Use the confposter theme supplied with this template

\setbeamercolor{block title}{fg=ngreen,bg=white} % Colors of the block titles
\setbeamercolor{block body}{fg=black,bg=white} % Colors of the body of blocks
\setbeamercolor{block alerted title}{fg=white,bg=dblue!50} % Colors of the highlighted block titles
\setbeamercolor{block alerted body}{fg=black,bg=dblue!10} % Colors of the body of highlighted blocks
% Many more colors are available for use in beamerthemeconfposter.sty

%-----------------------------------------------------------x   
% Define the column widths and overall poster size
% To set effecti ve sepwid, onecolwid and twocolwid values, first choose how many columns you want and how much separation you want between columns
% In this template, the separation width chosen is 0.024 of the paper width and a 4-column layout
% onecolwid should therefore be (1-(# of columns+1)*sepwid)/# of columns e.g. (1-(4+1)*0.024)/4 = 0.22
% Set twocolwid to be (2*onecolwid)+sepwid = 0.464
% Set threecolwid to be (3*onecolwid)+2*sepwid = 0.708

\newlength{\sepwid}
\newlength{\onecolwid}
\newlength{\twocolwid}
\newlength{\threecolwid}
\setlength{\paperwidth}{48in} % A0 width: 46.8in
\setlength{\paperheight}{36in} % A0 height: 33.1in
\setlength{\sepwid}{0.024\paperwidth} % Separation width (white space) between columns
\setlength{\onecolwid}{0.22\paperwidth} % Width of one column
\setlength{\twocolwid}{0.464\paperwidth} % Width of two columns
\setlength{\threecolwid}{0.708\paperwidth} % Width of three columns
\setlength{\topmargin}{-1.5in} % Reduce the top margin size
%-----------------------------------------------------------

\usepackage{graphicx}  % Required for including images

\usepackage{booktabs} % Top and bottom rules for tables

\usepackage{emoji}% Emoji :D
%-------------------------------------------s---------------------------------------------
%	TITLE SECTION 
%----------------------------------------------------------------------------------------

\title{Analysis of Bias Detection for English Newspapers {\emoji[ios]{1F4F0}}} % Poster title

\author{Curado, Antonio  {\emoji[ios]{1F9E0}}  Dahl, Morten} % Author(s)

\institute{Masters in Advanced Analytics @ Nova IMS} % Institution(s)

%----------------------------------------------------------------------------------------
%Elements:
%• abstract (1 2 sentences explaining your project)
%• description/intro (what is the problem? why is it hard? challenges?)
%• Methodology (how will you solve it)
%• methods you use (preprocessing you did, explain each method briefly, is it supervised/unsupervised, rule-based/ML, which assumptions )
%• metrics to evaluate you choose
%• results, highlighting what you want to show (make it visual / easy to understand) report metrics, videos or demos are a plus
%• show some examples of results
%• conclusions of results
%• future work: missing things from your work (work in progress) or future interesting work directions
%• make sure you cite every external code sources, datasets and papers you replicate/use (bibliography)
%• if you want and make the code publicly available write the link in the poster or QR code for people to use


\begin{document}

\addtobeamertemplate{block end}{}{\vspace*{2ex}} % White space under blocks
\addtobeamertemplate{block alerted end}{}{\vspace*{2ex}} % White space under highlighted (alert) blocks

\setlength{\belowcaptionskip}{2ex} % White space under figures
\setlength\belowdisplayshortskip{2ex} % White space under equations

\begin{frame}[t] % The whole poster is enclosed in one beamer frame

\begin{columns}[t] % The whole poster consists of three major columns, the second of which is split into two columns twice - the [t] option aligns each column's content to the top

\begin{column}{\sepwid}\end{column} % Empty spacer column

\begin{column}{\onecolwid} % The first column


%----------------------------------------------------------------------------------------
%	Abstract
%----------------------------------------------------------------------------------------

\begin{block}{Abstract}
This project contributes to the detection of fake news by analysing bias in english newspapers. These Articles are grouped by topic and been undertaken a sentiment analysis to detect the bias and tendencies within each article. It results in a visualization, which shows the media bias per newspaper, topic and keyword.

\end{block}


%----------------------------------------------------------------------------------------
%	Motivation
%----------------------------------------------------------------------------------------

\begin{block}{Motivation {\emoji[ios]{1F4AA}}}

    The motivation from this project lays in the recent doubts on media neutrality. People tend to distrust the media and call serious newspapers "fake news", whereas dubious newspapers gain popularity. So this project aims to provide the following:
    \begin{itemize}
    \item provide a measure for biasness and tendentious media coverage
    \item create transparency which newspapers tend to publish more tendentious articles
    \item initial notions towards an automated biasness detection in media coverage
    \end{itemize}

    \end{block}

%----------------------------------------------------------------------------------------
%	Methods
%----------------------------------------------------------------------------------------

\begin{block}{Methods}

The methodology followed in this project is as followed: As a first step data was aquiered from a \textbf{web scrapping} aproach from which news articles are downloaded from the web representations of various well-known newspapers. The data was limited to the politics section of each newspaper. It was tried to pick a balanced selection of different politically located newspapers through a qualitative assessment.
These articles were then filtered on keywords [e.g. "Trump", "Syria", "Brexit"] and been used to train a \textbf{LDA} model for topic detection. Simultaniously a \textbf{sentiment analysis} was carried out to measure the tendentious nature of each article. "A sentence on how it was done"
Finally the results of both analysis were mapped and visualised in an interactive scatterplot. 
\end{block}
    
%----------------------------------------------------------------------------------------

\end{column} % End of the first column

\begin{column}{\sepwid}\end{column} % Empty spacer column

\begin{column}{\onecolwid} % Begin a column which is two columns wide (column 2)


%----------------------------------------------------------------------------------------
%	MATERIALS
%----------------------------------------------------------------------------------------

\begin{block}{Data Extraction}

As a way to make the analysis relevant and up to date with the most current news topics
it has been developed a new news dataset with the following porperties

\begin{itemize}
\item Built a dataset with over \textbf{70.000} news articles
\item Scraped over \textbf{19} newspapers for over \textbf{2} weeks
\item With and average of \textbf{3.600} news articles per newspaper
\end{itemize}

\hfill\includegraphics{log_extraction.png}\hspace*{\fill}

The dataset was build only using the newspapers3k python package

\end{block}

%----------------------------------------------------------------------------------------
%	MATHEMATICAL SECTION
%----------------------------------------------------------------------------------------

\begin{block}{Dataset Properties}

Nam quis odio enim, in molestie libero. Vivamus cursus mi at nulla elementum sollicitudin. Nam quis odio enim, in molestie libero. Vivamus cursus mi at nulla elementum sollicitudin.
  
\begin{equation}
E = mc^{2}
\label{eqn:Einstein}
\end{equation}

Nam quis odio enim, in molestie libero. Vivamus cursus mi at nulla elementum sollicitudin. Nam quis odio enim, in molestie libero. Vivamus cursus mi at nulla elementum sollicitudin.

\begin{equation}
\cos^3 \theta =\frac{1}{4}\cos\theta+\frac{3}{4}\cos 3\theta
\label{eq:refname}
\end{equation}

Nam quis odio enim, in molestie libero. Vivamus cursus mi at nulla elementum sollicitudin. Nam quis odio enim, in molestie libero. Vivamus cursus mi at nulla elementum sollicitudin.

\begin{equation}
\kappa =\frac{\xi}{E_{\mathrm{max}}} %\mathbb{ZNR}
\end{equation}

\end{block}

%----------------------------------------------------------------------------------------

\end{column} % End of column 2



\begin{column}{\sepwid}\end{column} % Empty spacer column

\begin{column}{\twocolwid} % The third column

%----------------------------------------------------------------------------------------
%	CONCLUSION
%----------------------------------------------------------------------------------------

\begin{block}{Trump}

\begin{columns}[onlytextwidth]
    \begin{column}{.45\textwidth}
        \includegraphics[width=0.8\linewidth]{log_extraction.png} 
    \end{column}
    \begin{column}{.55\textwidth}
        Fusce quis massa dictum tortor \textbf{tincidunt mattis}. Donec quam est, lobortis quis pretium at, laoreet scelerisque lacus. Nam quis odio enim, in molestie libero. Vivamus cursus mi at \textit{nulla elementum sollicitudin}.
    \end{column}
\end{columns}

\end{block}

%----------------------------------------------------------------------------------------
%	ADDITIONAL INFORMATION
%----------------------------------------------------------------------------------------

\begin{block}{Brexit {\emoji[windows]{1F1E7}}}


    \begin{columns}[onlytextwidth]
        \begin{column}{.55\textwidth}
            Fusce quis massa dictum tortor \textbf{tincidunt mattis}. Donec quam est, lobortis quis pretium at, laoreet scelerisque lacus. Nam quis odio enim, in molestie libero. Vivamus cursus mi at \textit{nulla elementum sollicitudin}.
        \end{column}
        \begin{column}{.45\textwidth}
            \includegraphics[width=0.8\linewidth]{log_extraction.png} 
        \end{column}
    \end{columns}

\end{block}

%----------------------------------------------------------------------------------------
%	REFERENCES
%----------------------------------------------------------------------------------------

\begin{block}{Syria}
    \begin{columns}[onlytextwidth]
        \begin{column}{.45\textwidth}
            \includegraphics[width=0.8\linewidth]{log_extraction.png} 
        \end{column}
        \begin{column}{.55\textwidth}
            Fusce quis massa dictum tortor \textbf{tincidunt mattis}. Donec quam est, lobortis quis pretium at, laoreet scelerisque lacus. Nam quis odio enim, in molestie libero. Vivamus cursus mi at \textit{nulla elementum sollicitudin}.
        \end{column}
    \end{columns}
\end{block}

%----------------------------------------------------------------------------------------
%	FUTURE WORK
%----------------------------------------------------------------------------------------

\begin{block}{Future Work {\emoji[ios]{1F64F}}}

    \small{\rmfamily{Nam mollis tristique neque eu luctus. Suspendisse rutrum congue nisi sed convallis. Aenean id neque dolor. Pellentesque habitant morbi tristique senectus et netus et malesuada fames ac turpis egestas.}} \\
    
\end{block}

%----------------------------------------------------------------------------------------
%	ACKNOWLEDGEMENTS
%----------------------------------------------------------------------------------------

\setbeamercolor{block title}{fg=red,bg=white} % Change the block title color

\begin{block}{Acknowledgements {\emoji[ios]{1F64F}}}

\small{\rmfamily{Nam mollis tristique neque eu luctus. Suspendisse rutrum congue nisi sed convallis. Aenean id neque dolor. Pellentesque habitant morbi tristique senectus et netus et malesuada fames ac turpis egestas.}} \\

\end{block}

%----------------------------------------------------------------------------------------
%	REPO
%----------------------------------------------------------------------------------------

\begin{block}{ {\emoji[ios]{1F4BB}}}

    \small{\rmfamily{All code can be easily accessed in \href{https://github.com/morten-novaims/Text_Mining_HW}{github.com/morten-novaims/Text\_Mining\_HW}}} \\
    
    \end{block}

%----------------------------------------------------------------------------------------

\end{column} % End of the third column

\end{columns} % End of all the columns in the poster

\end{frame} % End of the enclosing frame

\end{document}